% !TEX program = xelatex
\documentclass[12pt, a4paper]{article}
\usepackage[utf8]{inputenc}

\usepackage{fontspec}
\setmainfont[Ligatures=TeX]{Linux Libertine O}

\usepackage[hidelinks, colorlinks = true, linkcolor = black, urlcolor = blue]{hyperref}
\usepackage{indentfirst}
\usepackage{graphicx}
\usepackage[left=1cm,right=1cm,top=2cm,bottom=2cm]{geometry}
\usepackage{lipsum}
\usepackage{caption}
\usepackage{subcaption}
\usepackage{dirtytalk}

\title{\textbf{Ηλεκτρονική 3} \\ \textbf{Εργασία Τελεστικού Ενισχυτή}}
\author{Θεόδωρος Κατζάλης \\ ΑΕΜ:9282 \\ katzalis@auth.gr}
\date{12 Ιανουάριου 2020}


\begin{document}

\maketitle
\sloppy
\tableofcontents
\pagebreak

\section{Εισαγωγή}

Ο τελεστικός ενισχυτής που θα επιχειρήσουμε να σχεδίασουμε είναι 2 σταδιών με είσοδο ΝMOS χώρις στάδιο εξόδου και με χωρητικό φορτίο. Αξίζει βέβαια να σημειωθεί ότι συνήθως προτιμάται η χρήση PMOS εισόδου εξαιτίας επίδρασης υποστρώματος κλπ.

Ο τελεστικός είναι πάντα 2 βαθμίδων; Γιατί χρησιμοποιήσες nmos, πότε το ένα, πότε το άλλο.

Οι προδιαγραφές με βάση την εκφώνηση της άσκησης παραμετροποιημένες με το ΑΕΜ είναι οι ακόλουθες:
\subsection{Αρχικές Συνθήκες}

%\vspace{1cm}

\begin{table}[h!]
\centering
\begin{tabular}{|c|c|}
	\hline
	Προδιαγραφές & AEM=9282  \\
	\hline
	\textbf{CL} & 2.82 (pF) \\
	\hline
	\textbf{SR} &  >18.82 (V/μs)\\
	\hline
	\textbf{Vdd} & 2.046 (V) \\
	\hline
	\textbf{Vss} & -2.046 (V)\\
	\hline
	\textbf{GB} & >7.82 (MHz)\\
    \hline
    \textbf{A} & >20.82 (dB)\\
    \hline
    \textbf{P} & <50.82 (mW)\\
    \hline
\end{tabular}
%\caption{}
\end{table}

\section{Περιγραφή αλγορίθμου (θεωρητική ανάλυση)}

Όσον αφορά το υπολογιστικό, θεωρητικό κομμάτι της σχεδίασης του τελεστικού ενισχυτή, ακολουθώντας τα βήματα απο τις σημειώσεις του μαθήματος, έχουμε να αναφερουμε τα εξής:

\subsection{Υπολογισμός σταθερών}

'Οσον αφορά την τάση κατωφλίου για τα MOS τραζίστορ, χρησιμοποιήσαμε τις τιμές που αναγράφονται στην περιγραφή των τρανζιστορ (KP), το οποίο θα μπορούσαμε να το υπολογίσουμε και χειροκίνητα με τον τύπο $C_{ox} = \mu_n \cdot \frac{\epsilon_{ox}}{t_{ox}}$, όπου  $\mu_n \equiv UO$ (spice model parameter). Πράγματι βλέπουμε ότι και με τους δύο τρόπους υπολογίζουμε τις ίδιες τιμές.
    
Επειδή δεν μας δίνεται κάποιο datasheet για τα συγκεκριμένα MOSFET προκειμένου να δούμε το έυρος του $V_{th}$ και τις μέγιστες και ελάχιστες τιμές, θεωρούμε ότι διατηρείται η τιμή σταθερή και ίση με αυτές που βρήκαμε απο την περιγραφή των μοντέλων. Δηλαδή $V_t(min) = V_t(max) = V_t$. Επίσης $V_{SB} = 0$ (zero body effect), οπότε $V_{to} = V_t$.

%Απο την άλλη ίσως θα μπορούσαμε να χρησιμοποιήσουμε το πινακάκι στις διαφάνεις +- 0.15

\subsection{Βήματα αλγορίθμου}    
\begin{enumerate}

    \item \textbf{Επιλογή τιμής L} (μήκος καναλιού). 
    
    Η τεχνολογία κατασκευής υποδηλώνει το ελάχιστο δυνατό μήκος καναλιού που μπορούμε να χρησιμοποποιήσουμε στην σχεδίαση μας και συνήθως επιλέγονται τιμές 1,5 ή 2 φορές αυτής της τιμής. Δεν μπορούμε βέβαια να χρησιμοποιήσουμε κάτι μικρότερο απο 0.35u. Για λόγους ευκολίας χρησιμοποιούμε την μονάδα.
    
    % Μatlab code here
    
    \item \textbf{Eπιλογή χωρητικότητας Miller}.
    
    Για να μην λειτουργεί ο ενισχυτής ως ταλαντωτής, σε συνδυασμό με την απόκλιση που μπορεί να έχει το κέρδος (βλέποντας την απόκριση συχνότητας), επίλεγεται phase margin 60 μοιρών, το οποίο απαιτεί να ικανοποιούνται οι εξής δύο συνθήκες: 
    \begin{itemize}
        \item $C_c > 0.22C_L$
        \item $g_{m6} > 10g_{m1}$
    \end{itemize}
    
    %Επιλέγουμε $C_c = 0.6204$ και δεν στρογγυλοποιούμε την τιμή, διότι αλλιώς δεν ικανοποιείται η δεύτερη συνθήκη, για την οποία θα γίνει λόγος στην συνέχεια.
    
    Επιλέγουμε $C_c = 0.6204\approx1$. 'Επειτα την στρογγυλοποίηση, όπως θα δούμε και στην συνέχεια, η δεύτερη συνθήκη δεν ισχύει για την δεδομένη τιμή $g_{m1}$, οπότε τελικά θα θεωρήσουμε $g_{m6} = 10g_{m1}$.
    
    \item Ρεύμα πόλωσης $Iref = I5 = I8$ ($I_7 \neq I_{ref}$, εξαιτίας συνθήκης για DC offset)
    
    \item Εύρεση $S_3 = (W/L)_3$. Βρίσκουμε $\leq1$, οπότε $S_3 = 1$.
    
    \item Έλεγχος $p_3 > 10 GB$.
    
    \item
\end{enumerate}

\subsection{Τελικές τιμές}

Πινακάκι για ρεύματα και W.

\section{Προσομοίωση SPICE}

Χρησιμοποιήσαμε για NMOS και PMOS τα μοντέλα mbreakN3 και mbreakP3 αντίστοιχα ($V_{SB} = 0$) και θέσαμε τις απαιτούμενες τιμές του κυκλώματος με αυτές του αλγρορίθμου. Μια πρώτη παρατήρηση σχετικά με τα ρεύματα είναι ότι $I_1 = I_2 = I_3 = I_4 = 10.19 \approx I_{ref}/2 = 9.94u$, η τιμής του αλγορίθμου. Βέβαια έχουμε μεγάλη απόκλιση για τα $I_6 = I_7 \approx 300u$, $50\%$ αύξηση. Θεωρούμε ότι ο αλγόριθμος μας είναι σωστός και συνεχίζουμε να δούμε αν καλύπτονται οι προδιαγραφές ή κατα πόσο αποκλίνουν.

\begin{figure}[h!]
	\centering
	\includegraphics[width = \textwidth, height = .4\textheight, keepaspectratio]{assets/base.png}
	\caption{Κύκλωμα προσομοίωσης τελεστικού ενισχυτή}
\end{figure}

\subsection{Έλεγχος προδιαγραφών}

Για τις ακόλουθες δύο προδιαγραφές κάνουμε AC sweep και προσθέτουμε τα κατάλληλα traces.

\subsubsection{A και GB}

\begin{figure}[h!]
	\centering
	\includegraphics[width = \textwidth, height = .25\textheight, keepaspectratio]{assets/gain_GB.png}
	%\caption{Κύκλωμα προσομοίωσης τελεστικού ενισχυτή}
\end{figure}

Βλέπουμε ότι έχουμε total dc gain ίσο με 24dB (15.84 V/V), το οποίο αποκίνει απο την θεωρητική τιμή (50dB), ωστόσο είναι πάνω απο την προδιαγραφή των 20.82dB.

Για να υπολογίσουμε το GB, βρίσκουμε την συχνότητα για την οποία το κέρδος έχει μειωθεί κατα 3dB. 'Οπως φαίνεται και απο το Probe Cursor, η τιμή της συχνότητας που αντιστοιχεί στα 21 dB είναι 363ΚHz. Συνεπώς:

\[  GB = \abs{A} \cdot f_H = 15.84 \cdot 363KHz \approx 5,74 < 7.82 MHz \]

Δεν πληρείται δηλαδή η προδιαγραφή για το Gain Bandwidth.

\subsubsection{Περιθώριο φάσης}

\begin{figure}[h!]
	\centering
	\includegraphics[width = \textwidth, height = .25\textheight, keepaspectratio]{assets/phase_margin.png}
	%\caption{Κύκλωμα προσομοίωσης τελεστικού ενισχυτή}
\end{figure}

Βλέπουμε ότι για 0 dB κέρδους, η τιμή της φάσης αντιστοιχεί σε $-99^o$. Οπότε έχουμε περιθώριο φάσης $180 - 99 = 91^ο$. Θα προσπαθήσουμε στη συνέχεια να το φτάσουμε κοντά στο 60. 

\pagebreak

\subsubsection{Slew rate}

Μετατρέπουμε την συνδεσμολογία σε συνδεσμολογία μοναδιαίου κέρδους βραχυκυκλώνοντας την αναστρέφουσα είσοδο με την έξοδο. Αλλάζουμε και την πηγή και έχουμε ως είσοδο τετραγωνικούς παλμούς.


\begin{figure}[h!]
     \begin{subfigure}[b]{0.5\textwidth}
         \centering
         \includegraphics[height=.4\textheight, width=\textwidth, keepaspectratio]{assets/slew_rate_circuit.png}
    \caption{V3 binary search}
     \end{subfigure}
     \begin{subfigure}[b]{0.5\textwidth}
         \centering
         \includegraphics[height=.4\textheight, width=\textwidth, keepaspectratio]{assets/slew_rate.png}
         \caption{V4 linear search} 
     \end{subfigure}
\end{figure}

Βρίσκουμε $18.441  \approx 18.82$ $MV/s$ $(V/\mu s)$. Δηλαδή για πολύ λίγο δεν πιάνουμε την προδιαγραφή.

\section{Tuning}

Όπως είδαμε προηφουμένως, οι τιμές για το περιθώριο φάσης, το εύρος κέρδους και τον ρυθμό ανόδου (slew rate) δεν ικανοποιούν τις προδιαγραφές. Παράλληλα θα προσπαθήσουμε να ενισχύσουμε και το κέρδος τάσης. Με βάση το ακόλουθο πινακάκι και παραμετρικές αναλύσεις καταλήξαμε στα εξής:

\begin{figure}[h!]
	\centering
	\includegraphics[width = \textwidth, height = .2\textheight, keepaspectratio]{assets/tuning_table.png}
	%\caption{Κύκλωμα προσομοίωσης τελεστικού ενισχυτή}
\end{figure}

Το κύκλωμα για το οποίο προκύπτουν τα ακόλουθα αποτελέσματα είναι το εξής:

\begin{figure}[h!]
	\centering
	\includegraphics[width = \textwidth, height = .4\textheight, keepaspectratio]{assets/circuit_tuned.png}
	%\caption{Κύκλωμα προσομοίωσης τελεστικού ενισχυτή}
\end{figure}


\subsection{Α και GB}

Για να αυξήσουμε το κέρδος πειραματιστήκαμε με το L αλλά και τον λόγο W/L των M1,2. Επίσης μειώνοντας το W του M7, είχαμε επίσης μια μικρή αύξηση του κέρδους (33db -> 36dB).

Τελικά, το κέρδος αυξήθηκε σε 36 dB (63 V/V) και η συχνότητα στην οποία μειώθηκε κατα 3dB είναι 158.41kHz. Οπότε: 

\[  GB = \abs{A} \cdot f_H = 63 \cdot 158.41KHz \approx 9.979 > 7.82 MHz \]

Πλέον πληρείται και η προδιαγραφή του GB.

\begin{figure}[h!]
	\centering
	\includegraphics[width = 0.8\textwidth, height = .3\textheight, keepaspectratio]{assets/gain_GB_tuned.png}
	%\caption{Κύκλωμα προσομοίωσης τελεστικού ενισχυτή}
\end{figure}

\subsection{Phase margin}

Για την βελτίωση του περιθωρίου φάσης εστιάσαμε την προσοχή μας στον πυκνωτή αντιστάθμισης Cc. Απο 1pF επιλέξαμε 0.7pF.

Βλέπουμε ότι για 0 dB κέρδους, η τιμή της φάσης αντιστοιχεί σε $-119.73^o$. Οπότε έχουμε περιθώριο φάσης $180 - 119.73 = 60.27^ο$.

\begin{figure}[h!]
	\centering
	\includegraphics[width = 0.8\textwidth, height = .3\textheight, keepaspectratio]{assets/phase_margin_tuned.png}
	%\caption{Κύκλωμα προσομοίωσης τελεστικού ενισχυτή}
\end{figure}

\subsection{Slew rate}

Για να αυξήσουμε το slew rate, αλλάξαμε τα W των Μ8,5. Επίσης η αλλαγή του πυκνωτή για να καλύπτεται η προδιαγραφή του περιθωρίου φάσης, οδήγησε επίσης στην βελτίωση του slew rate. Ακόμη παρατηρήσαμε ότι αυξάνοντας το ρεύμα πόλωσης υπήρχε μια μικρή βελτίωση του slew rate και μια πολύ μικρή μείωση του κέρδους.

\begin{figure}[h!]
	\centering
	\includegraphics[width = 0.8\textwidth, height = .3\textheight, keepaspectratio]{assets/slew_rate_tuned.png}
	%\caption{Κύκλωμα προσομοίωσης τελεστικού ενισχυτή}
\end{figure}

Βρίσκουμε $31.070 > 18.82$ $MV/s$ $(V/\mu s)$. 


%Μάλλον το ΑΕΜ και  η προσεκτική γραφή του αλγορίθμου, συνυπολογίζοντας συνθήκες όπως $g_{m6} \geq 10 \cdot g_{m1}$ και στρογγυλοποιώντας τους λόγους, οδηγηθήκαμε στο να μην χρειάζεται να κάνουμε tuning. Ωστόσο αξίζει να σημειωθεί ότι υπάρχει ένα πολύ βοηθητικό πινακάκι το οποίο ενδεχομένως θα μπορούσαμε να το χρησιμοποιήσουμε για να μεταβάλλουμε τις τιμές. Ενδεικτικά θα προσπαθήσουμε για παράδειγμα να αυξηθεί το κέρδος. 

%Φυσικά στο tuning θα πρέπει να μεταβάλουμε τα W με προσεκτικό τρόπο και να μην ξεχνάμε ότι υπάρχουν καθρέπτες ρεύματος για τους οποίους οι λόγοι θα πρέπει να είναι ταιριασμένοι.

\subsection{Ισχύς}

Για να υπολογίσουμε την ισχύ παίρνουμε απο την προσομοιώση τις τιμές των ρευμάτων και έχουμε:

\[ P_{diss} =  (I_6 + I_5) * (V_{dd} + V_{ss}) = (110.8u + 15u) * (2.046 + 2.046) = 500u = 0.5 mW < 50.82 mW\]

\pagebreak

\section{Πηγή widlar}

Σχεδιάζουμε το κύκλωμα πόλωσης και στόχος είναι να επιτύχουμε ένα ρεύμα 15$\mu$A. Αρχικά πειράξαμε λίγο το W του Μ14 για να χαμηλώσουμε το ρεύμα και έπειτα κάναμε μια παραμετρική ανάλυση για να βρούμε την κατάλληλη αντίσταση που θα μας δώσει 15$\mu$A. Τελικά βρήκαμε 4k.

\begin{figure}[h!]
	\centering
	\includegraphics[width = 0.8\textwidth, height = .3\textheight, keepaspectratio]{assets/widlar_circuit.png}
	%\caption{Κύκλωμα προσομοίωσης τελεστικού ενισχυτή}
\end{figure}

Τα αποτελέσματα με την πηγή είναι ακριβώς τα ίδια, μιας και πετύχαμε ίδιο ρεύμα πόλωσης.

\section{Παραμετρική ανάλυση θερμοκρασίας}

Απο $0-70^oC$ με βήμα 10 εκτελούμε τα παραπάνω χρησιμοποιώντας την πηγή widlar. Παρατηρούμε ότι όλες οι προδιαγραφές τηρούνται.

\subsection{A, GB και Phase margin}

\begin{figure}[h!]
	\centering
	\includegraphics[width = 0.8\textwidth, height = .3\textheight, keepaspectratio]{assets/widlar_all.png}
	%\caption{Κύκλωμα προσομοίωσης τελεστικού ενισχυτή}
\end{figure}

Ο κόκκινος cursor μας δείχνει που μηδενίζεται το κέρδος και ο πράσινος την συχνότητα που μειώνεται το κέρδος κατα 3dB.

\pagebreak
\subsection{Slew rate}

\begin{figure}[h!]
	\centering
	\includegraphics[width = 0.8\textwidth, height = .3\textheight, keepaspectratio]{assets/widlar_slew_rate.png}
	%\caption{Κύκλωμα προσομοίωσης τελεστικού ενισχυτή}
\end{figure}


\end{document}
