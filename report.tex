% !TEX program = xelatex
\documentclass[12pt, a4paper]{article}
\usepackage[utf8]{inputenc}

\usepackage{fontspec}
\setmainfont[Ligatures=TeX]{Linux Libertine O}

\usepackage[hidelinks, colorlinks = true, linkcolor = black, urlcolor = blue]{hyperref}
\usepackage{indentfirst}
\usepackage{graphicx}
\usepackage[left=1cm,right=1cm,top=2cm,bottom=2cm]{geometry}
\usepackage{lipsum}
\usepackage{caption}
\usepackage{subcaption}
\usepackage{dirtytalk}

\title{\textbf{Ηλεκτρονική 3} \\ \textbf{Εργασία Τελεστικού Ενισχυτή}}
\author{Θεόδωρος Κατζάλης \\ ΑΕΜ:9282 \\ katzalis@auth.gr}
\date{12 Ιανουάριου 2020}


\begin{document}
	
	\maketitle
	\sloppy
	\tableofcontents
	\pagebreak
	
	\section{Εισαγωγή}
	
	Ο τελεστικός ενισχυτής που θα επιχειρήσουμε να σχεδίασουμε θα έχει είσοδο ΝMOS. Αξίζει βέβαια να σημειωθεί ότι συνήθως προτιμάται η χρήση PMOS εισόδου για ...
	
	Ο τελεστικός είναι πάντα 2 βαθμίδων; Γιατί χρησιμοποιήσες nmos, πότε το ένα, πότε το άλλο.
	
	\section{Αρχικές Συνθήκες}
	
	%\vspace{1cm}
	
	\begin{table}[h!]
		\centering
		\begin{tabular}{|c|c|}
			\hline
			Προδιαγραφές & AEM=9282  \\
			\hline
			\textbf{CL} & 724 \\
			\hline
			\textbf{SR} & >709 \\
			\hline
			\textbf{Vdd} & 778 \\
			\hline
			\textbf{Vss} & -493 \\
			\hline
			\textbf{GB} & >493 \\
			\hline
			\textbf{A} & ?493 \\
			\hline
			\textbf{P} & 493 \\
			\hline
		\end{tabular}
		%\caption{Δείγματα Ιρλανδίας, Δείγμα Α}
	\end{table}
	
	\section{Περιγραφή αλγορίθμου}
	
	Ακολουθώντας τα βήματα του αλγορίθμου σχεδίασης τελεστικού ενισχυτή έχουμε τα εξής:
	\begin{enumerate}
		\item Επιλογή τιμής L (μήκος καναλιού). 
		
		Η τεχνολογία κατασκευής υποδηλώνει το ελάχιστο δυνατό μήκος καναλιού που μπορούμε να χρησιμποποιήσουμε στην σχεδίαση μας και συνήθως επιλέγονται τιμές 1,5 ή 2 φορές αυτής της τιμής. Δεν μπορούμε βέβαια να χρησιμοποιήσουμε κάτι μικρότερο απο 0.35u. Για λόγους ευκολίας χρησιμοποιούμε την μονάδα.
		
		% Μatlab code here
		
		\item Καθορισμός επιλογής χωρητικότητας Miller. Για λόγους ευστάθειας
		
		Προκύπτει . Επιλέγουμε.
		
		\item
	\end{enumerate}
\end{document}